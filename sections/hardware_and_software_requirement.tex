\newpage
\section{HARDWARE AND SOFTWARE REQUIREMENTS}
\subsection{Hardware Requirements}
The system should have sufficient hardware resource to support the application’s
processing and storage requirements. The minimum requirements are:
\begin{itemize}
    \item PC with Windows/Linux OS
    \item 8GB RAM 
    \item 2 GB Graphics card 
    \item 128 GB storage
    \item Processor: intel i5/Ryzen 5
\end{itemize}
\subsection{Software Requirements}
The software requirements for developing, training, evaluating, and deploying the
model effectively are listed below: 
\begin{enumerate}
    \item {\bf Integrated Development Environment (IDE):} An Integrated Development
    Environment (IDE) is a software application that provides comprehensive tools
    and features to streamline the development process. It typically includes a code
    editor, debugger, and other functionalities to make coding more efficient and
    productive. We will be using Visual Studio Code (VS Code) and Jupyter
    Notebook
    \item {\bf Version Control System:} Using version control, such as Git, will help to track
    changes, collaborate with others, and manage project effectively.
    \item {\bf Data Collection and Preprocessing Tools:} This consists of tools to collect and
    preprocess the training data. This includes web scraping library and tools to
    resize images.
    \item {\bf Web framework:} Web application framework simplify the process of building
    and deploying web applications, making it easier to integrate machine learning
    models and serve to users. We will be using either Django or Nodejs for
    backend and react for frontend.
    \item {\bf Deep-Learning Frameworks:}We are planning to use popular deep-learning
    frameworks like PyTorch or TensorFlow as per needed.
\end{enumerate}